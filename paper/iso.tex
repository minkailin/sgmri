\section{Isothermal limit}\label{isothermal}
We now examine the limit $\Gamma\to 1$, where the unperturbed disk becomes isothermal
, but perturbations are evolved with an adiabatic index
$\gamma=1.4$. We consider a nearly-isothermal polytropic background
and strictly isothermal backgrounds. These cases are treated
separately because the equilibrium structures have different
functional forms. A comparison between them provide another
check on our numerical results. 

\subsection{Large polytropic index}
We first consider setting $n=10$ to produce an almost radially
isothermal equilibrium with $p\propto \rho^{1.1}$. This allows us to use the
numerical code as set up for polytropic equilibria without
modification. We also adopt $A=2.5$ and $h=0.25$ for reasons given
in \S\ref{varpolyn}. The relatively large aspect-ratio does not
violate the thin-disk approximation as large $n$ implies the 
density decays rapidly away from the midplane. Also because of this,
we set the upper disk boundary at $Z_s=0.6$ to avoid very low 
densities. %All other parameters are the same as before. 

%3d-ness of core is 0.55
%3d-ness averaged over 0.9, 1.1 is 0.37
%%  Despite the larger
%% value of $\gamma/\Gamma$ than case 8 (see Table \ref{linsims}), the flow
%% three-dimensinality is the same.   

For this setup we obtained $\omega/m\Omega_0=0.9883$,
$\nu/\Omega_0=0.1375$ and $\avg{\theta_m}=0.35$. The top panel of
Fig. \ref{polyn10} shows the meridional flow at the vortex core. The
vortical motion is distinct and more apparent than case 3a, despite
the smaller value of $\gamma/\Gamma$ in the present case. 
However, apart from this difference, the solution is qualitatively
similar to case 3a.   

\begin{figure}[!t]
  \centering
  %% \includegraphics[scale=.425,clip=true,trim=0cm 1.cm 0cm
  %%   0cm]{figures/n10.0_gmma1.4_wq_3d_real_zslices.ps}
  \includegraphics[scale=.425,clip=true,trim=0cm 1.83cm 0cm
    0.25cm]{figures/n10.0_gmma1.4_vel3d_rz_dvz_nofill.ps}
   \includegraphics[scale=.425,clip=true,trim=0cm 0cm 0cm
    0.25cm]{figures/iso_vel3d_rz_dvz_nofill.ps}
  \caption{Perturbed meridional flow at $\phi=\phi_0$ for a $n=10$
    polytropic disk equilibrium (top) and a strictly isothermal
    equilibrium (bottom). 
    \label{polyn10}}
\end{figure}



\subsection{Strictly isothermal equilibrium}
%Because we have formulated the linear problem in terms of entropy and
%pressure length-scales, in order to calculate the RWI for strictly
%isothermal disks, we can just replace 
Modifications to our standard setup are required to treat  
disk equilibria with $p=\ciso^2\rho$ ($\Gamma\equiv 1$), where 
the constant sound speed $\ciso=\Hiso\Omega_k$, $\Hiso=\hiso
r_0(r/r_0)^{3/2}$ is the isothermal scale-height, and $\hiso$ is the
characteristic aspect-ratio at $r_0$. 
%We set $\ciso=\Hiso
%\Omega_k$, where $\Hiso\propto r^{3/2}$ is the isothermal scale-height
%and we define $\hiso=\Hiso/r_0$. 
The dimensionless vertical 
co-ordinate is now $Z=z/\Hiso$. The isothermal atmosphere is
exponential, $g(Z)=\exp{(-Z^2/2)}$, so there is no surface. In
practice we choose a finite vertical domain, i.e. $Z=Z_s$ represents a
constant number of isothermal scale-heights above the midplane. 

In the linear code we simply replace expressions for the entropy
and pressure length-scales by those corresponding to the isothermal
disk: the function $H\to\Hiso$ and $g(Z)$ becomes the Gaussian above. 
%% We set parameter values to mimic the large-$n$ polytrope in the 
%% previous section:  $Z_s=3$ and $\hiso=0.05$,
%% so that for both setups the pressure is reduced by roughly the 
%% same factor in going from the midplane to the upper boundary, and the
%% midplane temperature is roughly equal at $r_0$. 
{\bf
We choose $Z_s=3$ and $\hiso=0.05$, so the isothermal disk has roughly
the same temperature as that in the midplane of the large-$n$ 
polytrope considered above (at $r_0$).} In going from the
midplane to the upper boundary, the density is also reduced by
approximately the same factor for both cases.  


We obtain $\omega/m\Omega_0=0.9860$, $\nu/\Omega_0=0.1008$ and
$\avg{\theta_m}=0.39$. The perturbations plotted in
Fig. \ref{isothermal_case} are similar to case 3a, so we expect these
are features of the RWI in nonhomentropic flow, rather than
associated with the chosen parameter values. The perturbed meridional
flow shown in Fig. \ref{polyn10} (bottom panel) is in qualitative
agreement with the large-$n$ polytrope. 
The result is, however, quite different to
isothermal linear perturbations, for which \cite{meheut12} found the 
vertical velocity appears to have a node at $r_0$ ({\bf see their Fig. 3d where the vertical velocity changes sign across co-rotation radius}
, i.e. the fluid column is hydrostatic there). 
Note that both $\gamma/\Gamma$
and the growth rate are slightly smaller than the nonhomentropic case 3a,
but here the vortical motion is more prominent. 
%contribution from higher m?

\begin{figure}[!t]
  \centering
  \includegraphics[scale=.425,clip=true,trim=0cm 0.cm 0cm
    0cm]{figures/iso_wq_3d_real_zslices.ps}
  \caption{Pressure (top, $W$), density (middle, $Q$) and entropy
    (bottom, $S$) for a globally isothermal background. 
    \label{isothermal_case}}
\end{figure}

  Fig. \ref{polyn10_vz} shows the vertical 
  velocity at the vortex core as a function of height. The strictly
  isothermal background {\bf (thick solid)} has a slightly larger $\dd v_z$ than the
  large-$n$ polytrope {\bf (thick dashed)}. This is consistent 
  with previous findings that vertical motions oppose the RWI
  \citep{lin12c}, as the former case has a smaller growth rate than
  the latter. The {\bf thick lines are} qualitatively similar to case
  3a in Fig. \ref{compare_dvz}, but these are not directly comparable 
  because the present case differs in both the background structure and adiabatic index  to 
  those in Fig. \ref{compare_dvz}. 
  
  \begin{figure}[!t]
    \centering
    \includegraphics[scale=.425,clip=true,trim=0cm 0cm 0cm
      0.25cm]{figures/temp_dvz3}
    \caption{
      {\bf
        Vertical velocity as a function of $z$ at the vortex core
        $(r_0,\phi_0)$, for the $n=10$ 
        polytropic disk equilibrium (dashed) and a strictly isothermal
        equilibrium (solid) shown in 
        Fig. \ref{polyn10}, with free upper boundaries (thick lines). Corresponding thin 
        lines impose zero vertical velocity at $z=Z_s$ (growth rates increased by less than $0.5\%$ from the free boundary condition).
        Notice that changing upper disk boundary conditions only 
        affected the solution near $z=Z_s$ (cf. Fig. \ref{compare_dvz}). 
        This is consistent \cite{lin12c}, who found the
        influence of upper disk boundary condition to diminish with
        increasing polytropic index $n$. 
      }
      \label{polyn10_vz}}
\end{figure}
  
  Finally, we illustrate again a correlation 
  between meridional vortical flow and a tilted column of negative
  vertical vorticity perturbation in Fig. \ref{iso_vort3d_pz}. The figure is
  {\bf qualitatively} similar to that for polytropic backgrounds (case 3a in
  Fig. \ref{nonhomentropic_case_vort3d_pz}).{\bf  We find an average tilt
    of $1-~\avg{\cos{\theta}}_Z=0.0084
    \ll1$, so the vorticity column is nearly vertical.} 
  
  %% but the tilt magnitude
  %%   is larger for the isothermal background. 
  %Note that we chose
  %$r=1.03r_0$ as the radial slice, so that $\delta\omega_z <0$ along
  %the vertical line $(\phi_0,z)$. 
  %%  A plot for $r=1.02r_0$ also display
  %% the tilt, but in that case $\delta\omega_z$ was found to be positive
  %% at   
  

\begin{figure}[!t]
  \centering
  \includegraphics[scale=.425,clip=true,trim=0cm 0cm 0cm
    0.25cm]{figures/iso_vort3d_pz2.ps}
  \caption{
      Vertical vorticity perturbation, $\delta\omega_z$, in the
      $(\phi,z)$ plane at $r=1.03r_0$ for the strictly isothermal
      background. Regions of $\delta\omega\leq0$ are delineated by
      while lines. The center of meridional vortical motion identified
      in Fig. \ref{polyn10} occurs at height $z\sim
      H_\mathrm{iso}$. {\bf The azimuthal range
        $\phi-\phi_0\in[-0.5,0.5]\pi/m$ corresponds to anti-cyclonic
        motion about the vortex core.
        [ A plot for $r=1.02r_0$ also display
      tilted lines of constant $\delta\omega_z$, but in that case $\delta\omega_z > 0$
      at $(\phi_0, H_\mathrm{iso})$.]
      } 
    \label{iso_vort3d_pz}}
\end{figure}

\subsection{A nonlinear simulation}
%boundary conditions
%compare Q at different heights
%compare W in rz plane - vortical motion
We have also performed global 3D hydrodynamic simulations
using the \zeus  finite-difference code \citep{hays06}. As the focus
of this work is the linear problem, though, we defer a full discussion
of these nonlinear simulations to a follow-up paper. Our priority here
is to verify the  vortical motion in the meridional plane, which
appears characteristic in  the linear RWI solution for  nonhomentropic
flow.   

\subsubsection{Setup}
We use spherical polar co-ordinates $(\rsph,\theta,\phi)$ to describe
the disk, taken to be initially strictly isothermal as described above.  The
computational domain is $\rsph\in[0.2,2.0]r_0$,
$\theta\in[\theta_\mathrm{min},\pi/2]$, $\phi\in[0,2\pi]$ and is
divided into $(512,48,512)$ zones, with
$\tan{(\pi/2-\theta_\mathrm{min})}=3\hiso$ and $r_0=10$. 
The grid is logarithmically spaced in radius and uniformly spaced in 
the angular co-ordinates. Boundary conditions 
are outflow in $\rsph$, reflection in $\theta$ and periodic in $\phi$.  
Additional damping to meridional velocities near radial boundaries are
employed to reduce reflections \citep{valborro07}.  

%The disk is initially strictly isothermal as described in the previous section, and the vertical domain is such that 
%$\tan{(\pi/2-\theta_\mathrm{min})}=3\hiso$. 

After some experimentation, we found it was most convenient to start with a smooth disk.
In this case, a surface density $\Sigma\propto r^{-3/2}$, and tapered toward 
the inner boundary \citep[as used in][]{lin12b}. We introduce 
the density bump at $r=r_0$ via source terms in the mass, momentum and thermal energy equations, over a time-scale of $10P_0$, 
where $P_0 \equiv 2\pi/\Omega_k(r_0) $. %% is the Keplerian orbital period at $r_0$
This reduces numerical transients associated with initialization with {\bf a localized bump which has} large radial gradients. 
%At $t=10P_0$ we add small-amplitude random
%radial velocity perturbations. 

We choose the bump amplitude $A=1.25$ 
and isothermal aspect-ratio $\hiso=0.1$, as employed by \cite{meheut12} so that we can check our results against theirs.  
We measure perturbations with respect to azimuthally averaged hydrodynamic quantities at $t=10P_0$. 


\subsubsection{Results and comparison to linear flow}
%complex freq. averaged over r_sph\in[0.8,1.2]r_0 is
%<growth rate>/omega0=      0.18870212
%<pattern speed>/m.omega0=      0.98988571
%
%complex freq. averaged over r_sph\in[0.9,1.1]r_0
%<growth rate>/omega0=      0.19579173
%<pattern speed>/m.omega0=      0.98709079
%theta (3dness) is 0.34
We focus on the earliest stage of the instability, when perturbation
amplitudes are small so comparison with linear calculations can be
made. Fig. \ref{hydro_polar_dens} shows the snapshot to be
examined, {\bf taken at $t=23P_0$}. A $m=4$ mode has developed {\bf
  from numerical noise.} Notice the double-peak in
density perturbation, which is also present in
Fig. \ref{isothermal_case}. Using the method described in Appendix
\ref{instant_growth}, we estimated the $m=4$ mode growth rate and
frequency to be $\nu/\Omega_0\simeq 0.189$ and
$\omega/m\Omega_0\simeq 0.990$, in agreement with \cite{meheut12}.
%consistent with linear calculations of %\cite{meheut12}.  
Although they assumed barotropic perturbations, 
whereas we simulate adiabatic evolution, our linear calculations
indicate growth rates are largely unaffected by entropy gradients (Table \ref{linsims}). 

\begin{figure}[!t]
  \centering
  \includegraphics[scale=0.62,clip=true,trim=0cm 0.cm 0cm
    0cm]{figures/polar_dens015}
  \caption{Nonlinear hydrodynamic simulation of the RWI in a
    nonhomentropic 3D disk, initially isothermal but
    evolved adiabatically. The axes are in units of $r_0$. The relative density perturbation 
    near the midplane, scaled by 100, is shown. This quantity is
    proportional to the $Q$ used in linear calculations. The snapshot
    corresponds {\bf to the linear phase of the instability}. The drawn line   
    defines the vortex azimuth $\phi_0$ in Fig. \ref{lin_nonlin}---\ref{lin_nonlin_rz}.
    \label{hydro_polar_dens}}
\end{figure}

%use linear results for `big disk', theta_m is 0.39 
We have also computed this mode using the linear code as modified for
strictly isothermal equilibria, with a solid upper boundary. We obtain
growth rate and mode frequency $\nu/\Omega_0 = 0.1937$ and $\omega/m\Omega_0 = 
0.9896$, respectively. This is close to the nonlinear simulation. \cite{meheut12} suggested
smaller growth rate in the latter may be due to numerical viscosity. 
Fig. \ref{lin_nonlin} compares the density
perturbation $Q$ computed from the hydrodynamic simulation and linear code. They are broadly
consistent. The linear code also produces a bias toward the
over-density ahead of the vortex core at the midplane. Away from the
midplane, the center of the anti-cyclonic motion has shifted
downstream. 
This shows that, even within the linear regime, the vortex has
non-negligible vertical structure in the density perturbation (by comparing the two heights in Fig. \ref{lin_nonlin}). 
% for initially isothermal dis

\begin{figure}[!t]
  \centering
  \includegraphics[scale=.425,clip=true,trim=0cm 3.05cm 0cm
    1.3cm]{figures/streamxy_Q015_z0.0.ps}\includegraphics[scale=.425,clip=true,trim=2.2cm 
    3.05cm 0cm 1.3cm]{figures/iso_vel3d_rp_Q_HR_z0.0.ps}\\
  \includegraphics[scale=.425,clip=true,trim=0cm .0cm 0cm
    1.3cm]{figures/streamxy_Q015_z2.0.ps}\includegraphics[scale=.425,clip=true,trim=2.2cm 
    .0cm 0cm 1.3cm]{figures/iso_vel3d_rp_Q_HR_z2.0.ps}
  \caption{Normalized density perturbation, $Q$, associated with the
    RWI computed from a nonlinear hydrodynamic simulation (left) and
    the linear code (right), at the midplane (top) and at $2$
    scale-heights away from the midplane (bottom). The perturbed
    velocity field is also shown. The azimuthal wavenumber is $m=4$.  
  \label{lin_nonlin}}
\end{figure}

We compare meridional flows in Fig. \ref{lin_nonlin_rz}. The perturbed flow
is mostly horizontal in both cases.   
The nonlinear simulation also produce vortical motion in the same sense as
the linear calculation. For the \texttt{ZEUS} calculation, 
  we find the maximum magnitude of 
  vertical Mach number is $\sim 1\%$ with a density-weighted average
  value of $0.15\%$ in the shell $r_\mathrm{sph}\in[0.9,1.1]r_0$. 
The asymmetry of the pressure perturbation
about $r_0$ is captured by the linear code as well. Disagreement toward
the upper boundary is not unexpected, since the linear code assumes
the upper boundary is at a constant number of scale-heights above the
midplane, whereas the spherical grid imposes constant opening angle. 
However, both plots indicate $W$ increases away from the midplane in
the region exterior to $r_0$.  


\begin{figure}[!t]
  \centering
  \includegraphics[scale=.425,clip=true,trim=0cm 0.48cm 0cm
    0.4cm]{figures/hydro_vel3d_rz_W.ps}
  \includegraphics[scale=.425,clip=true,trim=0.0cm 
    .48cm 0.cm 0.4cm]{figures/iso_vel3d_rz_W_HR.ps}
  \caption{The perturbed velocity field projected onto the meridional
    plane at the vortex azimuth $\phi_0$, associated with the RWI
    calculated from a nonlinear hydrodynamic simulation (top) and the
    linear code (bottom). The average three-dimensionality, as
      measured by the ratio of vertical to meridional flow speeds,
      $\avg{\theta_m}$, is $0.39$ and $0.34$ in the linear and 
      nonlinear calculation, respectively. 
    A map of the normalized pressure
    perturbation is also shown. 
  \label{lin_nonlin_rz}}
\end{figure}






