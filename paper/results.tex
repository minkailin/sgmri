\section{Results}

\subsection{MRI in self-gravitating polytropic disks with uniform
  resistivity}  
We first calculate the MRI as a function of $Q$ in a polytropic disk
with uniform resistivity ($A=1$). Here, we consider $Q\in[0.2,4]$,
corresponding to stronly self-gravitating to non-self-gravitating
disks, for mid-plane Elsasser numbers $\Lambda_0\in[0.3,10]$. We fix $k_x=0.1$ and 
$\beta=100$. 

It is useful to first see how self-gravity may still play a role even
though density and potential perturbations are expected to be
negligible because $k_x\ll1$, i.e. the linear response is
non-self-gravitating. 
%even though the linear response is expected to be incompressible
%because $k_x\ll 1$, i.e. non-self-gravitating. 
\cite{sano99} found that the MRI wavelength, normalized by $H$, can be estimated as
\begin{align}
  \lambda=\mathrm{max}\left(\lambda_\mathrm{ideal},\lambda_\mathrm{resis}\right), 
\end{align}
where
\begin{align}
  \lambda_\mathrm{ideal} = f v_A = \frac{f}{\sqrt{\beta\rho}} 
\end{align}
is the ideal MRI wavelength, and 
\begin{align}
  \lambda_\mathrm{resis} =\frac{\eta}{f v_A}=\frac{f}{\Lambda_0}\sqrt{\frac{\rho}{\beta}}
\end{align}
is the MRI wavelength in the limit of high resistivity. 

\subsection{Resistive MRI in massive polytropic disks with layered
  resistivity} 


\subsection{GI in magnetized isothermal disks with uniform
  resistivity} 


\subsection{GI in magnetized isothermal disks with layered
  resistivity} 
