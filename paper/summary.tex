\section{Summary and discussion}\label{summary}
In this paper, we have performed axisymmetric linear stability
calculations of magnetized, self-gravitating disks in the local
approximation. Our models include resistivity and azimuthal fields.  
We have identified regimes under which the magneto-rotational
instability (MRI) is affected by disk self-gravity (SG).  
%what have we extended?

For a vertical field, the requirement for the MRI to operate is that its
vertical wavelength $\lambda \lesssim H$. While $\lambda$ is independent of
the strength of self-gravity $Q$, the disk thickness $H=H(Q)$ %lambda
%depends on omega, which might be
                                %lowered in a massive disk (sub-keplerian)
decreases with increasing SG, which reduces the MRI growth rates.   
%This
%effect becomes noticeable when the disk is marginally stable to GI in
%3D ($Q\sim 0.2$ or $Q_\mathrm{2D}\sim 0.7$).  
Thus, a sufficiently massive disk can surpress the MRI. %but GI may
                                %develop  instead
The MRI is also restricted to larger radial
scales as $Q$ is lowered. This means that the MRI becomes more global
in self-gravitating disks. Finally, the condition $\lambda < H$ 
requries $f(Q)/\sqrt{\beta}$ to be sufficiently small, so weaker
fields are required to permit the MRI with increasing self-gravity.   
 
%need larger boxes 
%This suggests
%that small-scale MRI turbulence may not develop in self-gravitating
%disks.      

%correlation

Interestingly, for layered resistivity we do not find layered
perturbations when the disk is massive. This is 
consistent with MRI becoming vertically global with increasing self-gravity.    
For non-self-gravitating disks $\lambda\ll H$, so the MRI can be
restricted to regions of size $L<H$, i.e. an active layer. This is not
possible if $\lambda \sim H$, as found for massive disks. Hence we
find magnetic perturbations penetrate into the high-resistivity dead
zone (e.g. $Q=0.2$ in Fig. \ref{poly_layer}), and there is no distinct
boundary between active and dead layers. This suggests that the
picture of layered accretion  \citep[e.g.][]{fleming03} may not be applicable 
to self-gravitating disks.  %strong magnetization 


%SG through 
We find MRI perturbations with radial scale of $\sim H$ can acquire  
density perturbations in a massive but Toomre-stable disk. 
This occurs when the MRI is weak, for example with a strong field or
high resisitivty. We argue in that case $\lambda\sim H$, so the density
perturbation caused by the MRI can be enhanced by self-gravity. 

%further enhancing density perturbations. 
At this point it is worth mentioning previous non-linear simulations
of magnetized self-gravitating galactic and circumstellar disks
\citep{kim03,fromang04a,fromang04}. These authors find self-gravity
does not enhance MRI density flucations significantly. However, they
employed ideal MHD simulations with gas-to-magnetic pressure ratios of
order $10^2$ to $10^3$. According  
to our results, self-gravity is not expected to affect the MRI in this
regime of $\beta$. On the other hand, \cite{fromang04} found
self-gravity makes MRI turbulence more coherent. This may be related
to our finding that small radial scale MRI is surpressed when
self-gravity is included in the background equilibria. 

Physically, we expect MRI to interact with self-gravity when
their spatial scales are similar. Because self-gravity acts globally in the
vertical direction, for it to affect the MRI, future non-linear
simluations should consider parameter regimes in which the MRI is
vertically global. Indeed, in the setup of \cite{kim03}, the disk
scale height exceeds the MRI vertical wavelength and self-gravity has
little impact.   

%We suggest that in
%order for self-gravity to be significant,
%We suggest that in
%order for self-gravity to influence MRI,      
%cite{kim03} and \cite{fromang04}
%Numerical experiments
%varying the field strength and resisitivity indicate this is again a
%result  

%GI/MRI: avoided crossings, relevance to simulations
Curiously, when GI and MRI are simultaneously supported, we find
unstable modes transition between MRI and GI. Such modes
have comparable potential and magnetic energy perturbations. These
transitions occur smoother with decreasing $\beta$
(Fig. \ref{compare_growth3}) or increasing $k_x$
(Fig. \ref{compare_growth3_Q01d2}). The latter implies that, in order
to capture magnetic-gravitational interactions represented by these
intermediate modes, non-linear simulations must resolve radial scales
smaller than the most unstable GI mode. For example, 
Fig. \ref{compare_growth3_Q01d2} suggest radial scales down to $\sim H/2$
should be resolved. 

%This is
%consistent with the MRI only affected by self-gravity for strong
%fields 
%mode classificication ambgious 
%This is consistent /expected
%numerical simulations should measure a continous range of modes
%This is
%consistent self-gravity only affecting the MRI when it is weak.  
%mag field only affect GI through pressure
%mri influenced by GI at large kx - > small scale -> need resolution

%azimuthal fields 

We examined the effect of an additional azimuthal field, while
keeping the vertical field at fixed strength. In this case, we also
relaxed the equitorial symmetry condition applied previously and
considered the full disk column. Self-gravity affects the MRI
differently depending on the midplane symmetry of the MRI density
perturbation. Self-gravity destabilizes MRI modes where the magnetic
energy reaches a minimum at $z=0$, which have a symmetric
density perturbation in the limit $B_y\to0$. However, 
self-gravity stabilizes MRI modes where the magnetic energy maximizes
at $z=0$. These modes have an anti-symmetric density
perturbation in the limit $B_y\to0$. This stabilization effect is
stronger for increasing $B_y$. Previous linear calculations indicate
increased compressibility associated with a toroidal field stabilizes
the MRI \citep{kim03}. We conjecture that self-gravity can
further enhance this effect.   
%can't do WKB (can't tell symmetries)

%lesure 14 (strong azimuthal field)
%implications non-linear MRI simulations with a strong toroidal field
%would over-estimate growth rates (and perhaps turbulence strengths)
%if SG is not accounted for. not applicable to self-gravitating disks     
%previous linear calcs show compressibility stabilizes MRI 
%additional pressure stabilizing MRI 

\subsection{Caveats and outlooks}
We discuss below two major extensions to our linear model that
should be undertaken, before embarking on non-linear simulations of
magnetized, self-gravitating disks, which is our eventual goal.  

\emph{Beyond the shearing box.} The shearing box ignores the curvature
of toroidal field lines present in the global disk 
geometry. \cite{pessah05} have shown that new effects on the MRI arise
when the curvature of a super-thermal toroidal field is accounted for;
although \citeauthor{pessah05} focused on modes with large (small)
vertical (radial) wavenumbers, for which we do not expect self-gravity
to be important. Since compressibilty becomes important for strong
toroidal fields, the effect of self-gravity may become even more
significant when super-thermal toroidal fields are
considered. However, global disk models will be neccessary
to self-consistently probe this regime. 

%
%This
%prevents consideration of super-thermal toroidal fields
%\citep{pessah05}.  


%tension stabilizes GI  
%This prevents us from considering super-thermal
%toroidal fields.  
%go to cylindrical disk geometry (fu & lai) 
%tension stabilizes GI 

%not looked at how MRI affects GI -> it doesn't because GI isn't
%concerned with shear. only magnetic pressure. maybe a plot of GI with
%magnetic perturbations? 

%spiral arms, GI angular mometum distribution 
%global modes 
%compressible term in y-induction equation 
\emph{Beyond axisymmetry.} Axisymmetric perturbations, as we have
assumed, preclude gravitational torques \citep{lynden-bell72}. 
%In 
%order to investigate how angular momentum transport 
The local non-axisymmetric stability of magnetized self-gravitating
thin disks has been considered by several authors
\citep{elmegreen87,gammie96b,fan97,kim01}. However, two-dimensional models
exclude the MRI. It will be neccessary to generalize these studies to
3D in order to investigate the impact of the MRI on angular momentum
transport by gravitational instability. Furthermore, self-gravitating
disks can develop global spiral instabilities while stable against local 
axisymmetric perturbations \citep{papaloizou89,papaloizou91}. Global
non-axisymmetric linear models will be desirable to support non-linar
simulations of this kind \citep{fromang04c,fromang05}.   



%Our linear problem assumes a non-zero
%vertical field is always present. It is of interest to 
%consider the opposite limit of purely toroidal fields. Of course, one
%must consider the non-axisymmetric MRI \cite{}
%The local non-axisymmetric stability of magnetized, self-gravitating
%disks has been treated in the shearing sheet framework \citep{} 
%framework by decomposing
%perturbations into shearing waves %ref.  
%
%global angular momentum exchange
%It is necessary to generalize the current models to non-axisymmetric
%perturbations. 
%It would be interesting to examine global spiral instabilities in a 
%magnetized disk. 
%In particular, to examine global spiral instabilities
%in a magnetized disk. 
%fu lai 


