\section{Summary and discussion}\label{summary}
In this paper, we have performed explicit stability calculations of
magnetized, self-gravitating disks in the local approximation. 



%layered accretion incompatable in massive disks

%why did previous MRI simulations find no difference between SG and
%nonSG? prob because the mri mode has too small vertical wavelength


\subsection{Caveats and outlooks}
We discuss below two major extensions to our linear model that
should be undertaken, before performing non-linear simulations of
magnetized, self-gravitating disks, which is our eventual goal.  

\emph{Beyond the shearing box.} The shearing box ignores the curvature
of toroidal field lines present in the global disk
geometry. \cite{pessah05} have shown that new effects on the MRI arise
when the curvature of a super-thermal toroidal field is accounted for;
although \citeauthor{pessah05} focused on modes with large (small)
vertical (radial) wavenumbers, for which we do not expect self-gravity
to be important. Since compressibilty becomes important for strong
toroidal fields, the effect of self-gravity may become even more
significant when super-thermal toroidal fields are
considered. However, global disk models will be neccessary
to self-consistently probe this regime. 

%
%This
%prevents consideration of super-thermal toroidal fields
%\citep{pessah05}.  


%tension stabilizes GI  
%This prevents us from considering super-thermal
%toroidal fields.  
%go to cylindrical disk geometry (fu & lai) 
%tension stabilizes GI 


%spiral arms, GI angular mometum distribution 
\emph{Beyond axisymmetry.} It is well known that self-gravitating
disks can be unstable to non-axisymmetric perturbations but stable
against axisymmetric perturbations, although the threshold Toomre
parameter for both is still of order unity. %papaloizou/lin,
                                %sajvoini??
 


