\section{Summary and discussion}\label{summary}
In this paper, we have performed axisymmetric linear stability
calculations of magnetized, self-gravitating, vertically stratified
disks in the local approximation. Our models include resistivity and
azimuthal fields. We have identified regimes under which the
magneto-rotational instability (MRI) is affected by disk self-gravity
(SG).   


For a vertical field, the requirement for the MRI is that   
its vertical wavelength $\lambda \lesssim H$. While $\lambda$ is independent of
the strength of self-gravity $Q$, the disk thickness $H=H(Q)$ 
decreases with increasing SG, which can reducie MRI growth rates. 
Thus, a sufficiently massive disk can potentially surpress the MRI. 
The MRI is also restricted to larger radial
scales as $Q$ is lowered. This means that the MRI becomes more global
in self-gravitating disks. Finally, the condition $\lambda < H$ 
requries $f(Q)/\sqrt{\beta}$ to be sufficiently small, so weaker
fields are required to permit the MRI with increasing self-gravity.   
%since MRI is 3D, strong SG makes it 2D 
 
Interestingly, for layered resistivity we do not find layered
magnetic perturbations when the disk is massive. This is 
consistent with the MRI becoming vertically global with increasing self-gravity.    
For non-self-gravitating disks $\lambda\ll H$, so the MRI can be
restricted to regions of size $L<H$, i.e. an active layer. This not
compatible with $\lambda \sim H$, as found for massive disks. Hence we
find magnetic perturbations penetrate into the high-resistivity dead
zone (e.g. $Q=0.2$ in Fig. \ref{poly_layer}), and there is no distinct
boundary between active and dead layers. This suggests that the
picture of layered accretion  \citep[e.g.][]{fleming03} may not be applicable 
to self-gravitating disks.  

%SG through 
We find MRI modes with radial scales of $\sim H$ can acquire  
density perturbations in massive but Toomre-stable disks. 
This occurs when the MRI is weak, for example with a strong field or
high resisitivty. We argue in that case $\lambda\sim H$, so the MRI is
compressible and the associated density
perturbation can be enhanced by self-gravity.
%Our 
%interpretation involves the MRI to provide a seed density
%perturbation %in the absence of self-gravity. 

%what is required? a seed pert, plus amplification

%further enhancing density perturbations. 
At this point it is worth mentioning previous non-linear simulations 
of magnetized self-gravitating galactic and circumstellar disks
\citep{kim03,fromang04a,fromang04}. These authors find self-gravity
did not enhance MRI density flucations significantly. However, they
employed ideal MHD simulations with gas-to-magnetic pressure ratios of
order $10^2$ to $10^3$. %MRI is basically incompressible 
This is qualitatively consistent with our results, as 
self-gravity is not expected to influence the MRI in this  
regime of $\beta$, except through the background state. 
%both `seed' and amplfication are small 
For example, \cite{fromang04} found
MRI turbulence is more coherent in self-gravitating disks. This may be related
to our finding that small radial scale MRI is surpressed when
self-gravity is included in the background equilibria. 

Physically, we expect MRI to interact with self-gravity when
their spatial scales are similar. Because self-gravity acts globally in the
vertical direction, for it to affect the MRI, future non-linear
simluations should consider parameter regimes in which the MRI is
vertically global. %this makes MRI more compressible
Indeed, in the setup of \cite{kim03}, the disk
scale height exceeds the MRI vertical wavelength and self-gravity has
little impact.   



%GI/MRI: avoided crossings, relevance to simulations
Curiously, when GI and MRI are simultaneously supported, we find
unstable modes transition between MRI and GI. There exists modes 
with comparable potential and magnetic energy perturbations, which 
demonstrates MRI and GI can interact. These 
transitions occur smoother with decreasing $\beta$
(Fig. \ref{compare_growth3}) or increasing $k_x$
(Fig. \ref{compare_growth3_Q01d2}). The latter implies that, in order
to capture the magneto-gravitational interactions represented by these
intermediate modes, non-linear simulations must resolve radial scales
smaller than the most unstable GI mode. For example, 
Fig. \ref{compare_growth3_Q01d2} suggest radial scales down to $\sim H/2$
should be well-resolved. 

%azimuthal fields 

We examined the effect of an additional azimuthal field, while
keeping the vertical field at fixed strength. In this case, we also
relaxed the equitorial symmetry condition applied previously and
considered the full disk column. Self-gravity affects the MRI
differently depending on its character. 
%midplane symmetry of the MRI 
%density
%perturbation. 
Self-gravity destabilizes MRI modes where the magnetic
energy has a minimum at $z=0$, these modes have a symmetric
density perturbation in the limit $B_y\to0$. However, 
self-gravity stabilizes MRI modes where the magnetic energy has a
maximum at $z=0$, these modes have an anti-symmetric density
perturbation in the limit $B_y\to0$. This stabilization effect is
stronger for increasing $B_y$. Previous linear calculations show that 
increased compressibility associated with a toroidal field stabilizes
the MRI \citep{kim03}. We conjecture that self-gravity 
further enhances this effect. Non-linear MRI simulations with strong
toroidal fields that neglect self-gravity may over-estimate the
strength of MRI turbulence.   
%can't do WKB (can't tell symmetries)



\subsection{Caveats and outlooks}
We discuss below two major extensions to our linear model that
should be undertaken, before embarking on non-linear simulations of
magnetized, self-gravitating disks, which is our eventual goal.  

\emph{Beyond the shearing box.} The shearing box ignores the curvature
of toroidal field lines present in the global disk 
geometry. \cite{pessah05} demonstrated new effects on the MRI 
when the curvature of a super-thermal toroidal field is accounted for;
although \citeauthor{pessah05} focused on modes with large (small)
vertical (radial) wavenumbers, for which we expect self-gravity
can be ignored. 
Since compressibilty becomes important for strong
toroidal fields, the effect of self-gravity on modes with $k_xH\sim1$
may become significant when super-thermal toroidal fields are
considered. However, global disk models will be neccessary
to self-consistently probe this regime. 


\emph{Beyond axisymmetry.} Axisymmetric perturbations, as we have
assumed, preclude gravitational torques \citep{lynden-bell72}. 
The local non-axisymmetric stability of magnetized self-gravitating
thin disks has been considered by several authors
\citep{elmegreen87,gammie96b,fan97,kim01}. However, two-dimensional models
exclude the MRI. It will be neccessary to generalize these studies to
3D in order to investigate the impact of the MRI on angular momentum
transport by gravitational instability. Furthermore, self-gravitating
disks can develop global spiral instabilities while stable against local 
axisymmetric perturbations \citep{papaloizou89,papaloizou91}. Global
non-axisymmetric linear models will be desirable to support non-linar
simulations of this kind \citep{fromang04c,fromang05}.    

