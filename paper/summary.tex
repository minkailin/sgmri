\section{Summary and discussion}\label{summary}
In this paper, we have examined the linear stability of radially
structured three-dimensional disks with non-uniform entropy
distribution. These calculations may be considered as an extension to 
the 2D Rossby wave instability \citep{li00} by adding the vertical
dimension, or to the barotropic RWI calculations of \citetalias{lin12} by
adding an energy equation with a simpler numerical method. 

We adopted polytropic disk equilibria so that
the magnitude of entropy gradients can be conveniently parametrized
by $\Delta\gamma\equiv\gamma/\Gamma$, and we focused on the effect of
$\Delta\gamma\geq1$. When the background density and
velocity field is fixed through $\Gamma$, we found increasing $\Delta
\gamma$ has negligible effect on the instability growth rate. However,
{\bf the magnitude of} pressure and density perturbations increase with height, and the 
meridional flow associated with the vortex core is qualitatively
changed, with the introduction of {\bf meridional} vortical
motion. 

{\bf Meridional vortical motion was found to correlate with a small 
  tilt of a fluid column with negative vertical vorticity
  perturbation.  In standard hydrodynamics, vorticity tilting
  can originate from a contribution of the
  $\bm{\omega}\cdot\nabla\bm{v}$ term in the evolution equation of the
  vorticity independently of the baroclinic source term $\nabla \rho
  \times\nabla p$. However, given the tilt is absent in our homentropic
  calculations, we associate the tilt with the baroclinic source term, which
  produces azimuthal vorticity. } 
We also found
that the vertical velocity at the vortex core is no longer linear in
$z$, as for homentropic flow. 

In our second set of experiments, we fixed $\gamma$ and decreased
$\Gamma$. We found that by making the flow nonhomentropic, the
co-rotation region became \emph{more} three-dimensional, despite the
decrease in growth rate. This result
is opposite to \citetalias{lin12} where lowering $\Gamma$ made the
flow less three-dimensional. This implies that entropy gradients  
play an important role in the vertical structure of the perturbations. 
%This will likely have impact
%on the motion of dust particles within Rossby vortices, in particular 
%as the flow is dominated by bouyancy forces away from the midplane. 

We also considered isothermal equilibria. A linear calculation with 
$\Gamma=1.1$ and one with a strictly isothermal setup ($\Gamma\equiv
1$) were consistent. Both produced prominent meridional 
vortical motion. In order to verify this feature, we ran a nonlinear
simulation of the RWI in an initially isothermal disk, but evolved
adiabatically. We indeed identified said  vortical motion.    
%although the agreement with linear calculations is
%qualitative. This might imply that nonlinear effects become important
%for the RWI vertical structure earlier than in the horizontal
%direction. 
Keeping in mind that the setup for linear and nonlinear simulations 
were not identical (e.g. numerical grid, boundary treatment), similarities
between them, such as mode frequency, growth rate and horizontal flow,
are satisfactory.  

Vortical motion in the meridional plane thus appears characteristic of 
the \emph{linear} RWI in nonhomentropic disks. Whether or not this is 
significant for the vortex evolution can only be answered by detailed 
long term nonlinear simulations. If this vortical motion is present 
in the nonlinear regime then it may prevent dust particles from reaching the disk surface, 
which occurs for homentropic flow \citep{meheut12c}. 

However, given this meridional vortical motion is absent in the homentropic 
linear solution, it may eventually vanish because of entropy
mixing, if no mechanism is present to maintain entropy gradients. 
For example, the background entropy increases with height but 
the linear entropy perturbation becomes more negative with height, {\bf 
and its magnitude grows exponentially in time}. 
{\bf Indeed,  recent 3D fully compressible simulations in
  nonhomentropic disks shows that well into the nonlinear regime,
  Rossby vortices are columnar
  \citep{richard13}}. On the other hand, \cite{meheut12} observed
strong meridional vortical motion in their homentropic hydrodynamic
simulations; we conclude they are of nonlinear origin.  

In the linear solutions, we often observe perturbation magnitudes
increase away from the midplane in nonhomentropic
disks\footnote{This reminds us of the off-midplane vortices
  discovered by \cite{barranco05} in nonlinear local simulations, 
%  which suggest the non-zero vertical bouyancy frequency to be
%  stabilizing   against secondary instabilities \citep{lesur09}. 
  but the  
  setup considered in that study is very different from the
  present work. Nevertheless, in both cases the vertical entropy
  gradient is stabilizing away from the midplane. 
  %%  Nevertheless, a positive vertical bouyancy frequency
  %% may stabilize 3D vortices \citep{lesur09}.
} 
(e.g. Fig. \ref{lin_nonlin_rz}). Then the RWI may not be as robust
against vertical boundary conditions as it is to radial boundary
conditions. This could pose difficulty for the RWI to develop in dead
zones of real protoplanetary disks, which are expected to be confined
from above and below by magnetically turbulent layers
\citep{oishi09}. The vertical boundary condition set by these layers
may or may not be compatible with the linear RWI solution.    


\subsection{Caveats and outlooks}
%need a good guess
One trade-off for the simplicity of our numerical method for 
linear simulations is that a trial eigenfrequency must be
guessed. This is not a significant obstacle for the problem at
hand, because previous RWI studies provide an important guide
\citep[][]{li00}. Otherwise, zeros of the complex function
$\mathcal{D}(\sigma)=\mathrm{det}\,\bm{U}$ need to be located with more 
rigorous methods \citep[e.g.][]{kojima86,valborro07}. We have also exploited
previous findings that the PPI and RWI are predominantly two-dimensional 
\citep{papaloizou85, goldreich86, kojima89, umurhan10, meheut12, lin12, lin12c}, which
enabled the use of a small number of basis functions. 
However, there could exist parameter regimes where the RWI
has significant vertical structure, rendering our solution method
inefficient. 

%extension to baroclinic equilibria 
%simple polytrope background
Our conclusions are limited to polytropic backgrounds. While this was 
convenient for numerical experiments, it is an over-simplification
of protoplanetary disks, which are 
expected to have complicated vertical structure \citep{terquem08}. In
particular, we found that entropy gradients plays a role in the
vertical structure of the {\bf linear} RWI, and even a small entropy gradient can noticeably
modify the vertical flow (\S\ref{nz_nonzero}).
Thus, a realistic model for entropy evolution is needed. 
%This is important because even a small entropy gradient can noticeably 
%modify the vertical flow (\S\ref{nz_nonzero}). 

It would also be of interest to generalize the calculations
to \emph{baroclinic} equilibria\footnote{In fact, baroclinic tori were
  briefly considered by \cite{frank88}.}, 
for which $\p_z\Omega\neq0$. This may well be the 
case when the equilibrium pressure depends on both the density and temperature. 
%This requires 
%derivation of the linearized equations including terms involving $\p_z\Omega$. 
Complications from baroclinic instabilities may arise, however \citep{knobloch86, umurhan12, nelson12}.
%% This may involve adding heating or cooling
%% sources to the energy equation. 
%source terms - friction, cooling
%new physics - self-gravity

We have neglected gas self-gravity in this study. Our models therefore
assume that the Toomre parameter is much larger than unity in both the 
unperturbed \emph{and} perturbed states. {\bf However, self-gravity
  may affect the RWI even when the Toomre parameter is not small
  \citep{lovelace12}}. 
Previous studies
have found  higher $m$ RWI modes are favored when disk self-gravity is
included \citep{lyra08,lin11a}. Recent 3D simulations of the RWI in a
locally isothermal disk show that vertical self-gravity can noticeably
enhance the density perturbation near the midplane,
even though the initial disk was considered low mass \citep{lin12b}. %%  The effects
%%  of self-gravity %% should be considered in conjunction with disk
%%  thermodynamics. %In a nonlinear simulation, the latter may not hold even if the former
%does.
%Previous studies of the RWI also find higher $m$ modes modes are
%favored when disk self-gravity is included
%\citep{lyra08,lin11a}.    

In principle, one can express the Poisson integral as a matrix
operator and  incorporate it into our formalism. {\bf The linear} problem is
further complicated by the need of a numerical solution to the
equilibrium equations describing a radially structured,
self-gravitating 3D disk \citep{muto11}.  Such a linear calculation is
beyond the scope of this paper, but will be inevitable for
understanding  the RWI in 3D self-gravitating disks. Perhaps a simpler
starting point, to gain first insight, is  direct hydrodynamic 
simulations {\bf including disk gravity}. This is indeed the approach taken in our follow-up paper.    
%\clearpage
%since the ZEUS-MP code already includes a Poisson solver (as used in Lin 2012b). 

%have shown that
%self-gravity reduces the RWI growth rate through the potential perturbation, 
%but may be destabilizing through modification of the background disk \citep{lin11a}. 
