\section{Summary and discussion}\label{summary}
In this paper, we have performed axisymmetric linear stability
calculations of magnetized, self-gravitating disks in the local
approximation. Our models include resistivity and azimuthal fields.  
We have identified regimes under which the magneto-rotational
instability (MRI) is affected by disk self-gravity (SG).  
%what have we extended?

For a vertical field, the requirement for the MRI to operate is that its
vertical wavelength $\lambda \lesssim H$. While $\lambda$ is independent of
the strength of self-gravity $Q$, the disk thickness $H=H(Q)$ %lambda
%depends on omega, which might be
                                %lowered in a massive disk (sub-keplerian)
decreases with increasing SG, which reduces the MRI growth rates.   
%This
%effect becomes noticeable when the disk is marginally stable to GI in
%3D ($Q\sim 0.2$ or $Q_\mathrm{2D}\sim 0.7$).  
Thus, a sufficiently massive disk can surpress the MRI. %but GI may
                                %develop  instead
The MRI is also restricted to larger radial
scales as $Q$ is lowered. This means that the MRI becomes more global
in self-gravitating disks. Finally, the condition $\lambda < H$ 
requries $f(Q)/\sqrt{\beta}$ to be sufficiently small, so weaker
fields are required to permit the MRI with increasing self-gravity.   
 
%need larger boxes 
%This suggests
%that small-scale MRI turbulence may not develop in self-gravitating
%disks.      

%correlation

Interestingly, for layered resistivity we do not find layered
perturbations when the disk is massive. This is 
consistent with MRI becoming vertically global with increasing self-gravity.    
For non-self-gravitating disks $\lambda\ll H$, so the MRI can be
restricted to regions of size $L<H$, i.e. an active layer. This is not
possible if $\lambda \sim H$, as found for massive disks. Hence we
find magnetic perturbations penetrate into the high-resistivity dead
zone (e.g. $Q=0.2$ in Fig. \ref{poly_layer}), and there is no distinct
boundary between active and dead layers. This suggests that the
picture of layered accretion  \citep[e.g.][]{fleming03} may not be applicable 
to self-gravitating disks.  %strong magnetization 


%SG through 
We find MRI perturbations with radial scale of $\sim H$ can acquire  
density perturbations in a massive but Toomre-stable disk. 
This occurs when the MRI is weak, for example with a strong field or
high resisitivty. We argue in that case $\lambda\simH$, so the density
perturbation caused by the MRI can be enhanced by self-gravity. 

%further enhancing density perturbations. 
At this point it is worth mentioning previous non-linear calculations 
performed by \cite{kim03} and \cite{fromang04a,fromang04} in the
context of galactic and circumstellar disks, respectively. These
authors simulated the MRI in self-gravitating disks. They find
self-gravity has little impact on the MRI.   



%cite{kim03} and \cite{fromang04}



%Numerical experiments
%varying the field strength and resisitivity indicate this is again a
%result  





%GI/MRI: avoided crossings 

%azimuthal fields 

%\subsection{Implications for non-linear simulations}
%what scaes? resolution
%why don't formang  and kim/ostriker see any difference? 

\subsection{Caveats and outlooks}
We discuss below two major extensions to our linear model that
should be undertaken, before performing non-linear simulations of
magnetized, self-gravitating disks, which is our eventual goal.  

\emph{Beyond the shearing box.} The shearing box ignores the curvature
of toroidal field lines present in the global disk
geometry. \cite{pessah05} have shown that new effects on the MRI arise
when the curvature of a super-thermal toroidal field is accounted for;
although \citeauthor{pessah05} focused on modes with large (small)
vertical (radial) wavenumbers, for which we do not expect self-gravity
to be important. Since compressibilty becomes important for strong
toroidal fields, the effect of self-gravity may become even more
significant when super-thermal toroidal fields are
considered. However, global disk models will be neccessary
to self-consistently probe this regime. 

%
%This
%prevents consideration of super-thermal toroidal fields
%\citep{pessah05}.  


%tension stabilizes GI  
%This prevents us from considering super-thermal
%toroidal fields.  
%go to cylindrical disk geometry (fu & lai) 
%tension stabilizes GI 


%spiral arms, GI angular mometum distribution 
\emph{Beyond axisymmetry.} It is well known that self-gravitating
disks can be unstable to non-axisymmetric perturbations but stable
against axisymmetric perturbations, although the threshold Toomre
parameter for both is still of order unity. %papaloizou/lin,
                                %sajvoini??
 


